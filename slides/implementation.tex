\section{Implementation}
\begin{frame}[t]
\sectionpage\vskip 18pt
    \begin{itemize}
        \item<1-> Questions to be answered:\vskip 18pt
            \begin{enumerate}
                \item<2-> Why are rSS without contrastive stress infelicitous?\vskip 9pt
                \item<2-> How does contrastive stress help?\vskip 9pt
                \item<2-> Why contrastive stress on auxiliary verb?\vskip 9pt
                \item<3-> How does a causal link between {\color{red}$\phi$} and {\color{Orange}$\psi$} cause infelicity?\vskip 9pt
                \item<3-> How does non-counterfactuality cause infelicity?\vskip 9pt
                \item<4-> How does the epistemic dismissal of {\color{Orange}$\psi$} rescue rSS?
            \end{enumerate}
	\end{itemize}
\end{frame}

\subsection{Why are rSS without contrastive stress infelicitous?}
\begin{frame}[t]
    \subsectionpage\vskip 9pt
    \begin{itemize}
        \item<1-> rSS without contrastive stress are routinely infelicitous\vskip 9pt
        \item<2-> Requirement: Some ingredient renders rSS infelicitous without further intervention\vskip 4.5pt
            \begin{itemize}
                \item<3-> Infelicity via contradiction: Extend {\color{Orange}$\psi$} into the {\color{red}$\phi$}-conditional
                \item<4-> Dynamic Strict Semantics: Expanding Modal Horizon
                \item<5-> Variably-Strict Semantics: Modal Subordination \citep{Klecha2014a}
            \end{itemize}\vskip 18pt
        \item<6-> Modal Subordination: interpret current sentence as subordinate to preceding modal\vskip 4.5pt
            \begin{itemize}
                \item<7-> Second consequent interpreted w.r.t. both its and the preceding antecedent
            \end{itemize}
    \end{itemize}
    \visible<8->{\ex. \resizebox{576pt}{!}{If $({\color{red}\phi}\land{\color{Orange}\psi})$ then not $\color{OliveGreen}\chi$; but if $\color{red}\phi$ then $\color{OliveGreen}\chi$ $\Rightarrow$ If $({\color{red}\phi}\land{\color{Orange}\psi})$ then not $\color{OliveGreen}\chi$; but if $({\color{red}\phi}\land{\color{Orange}\psi})\land\color{red}\phi$ then $\color{OliveGreen}\chi$}\\%
    \hspace{-0.75mm}\resizebox{576pt}{!}{\phantom{If $({\color{red}\phi}\land{\color{Orange}\psi})$ then not $\color{OliveGreen}\chi$; but if $\color{red}\phi$ then $\color{OliveGreen}\chi$ }$\Rightarrow$ If $({\color{red}\phi}\land{\color{Orange}\psi})$ then not $\color{OliveGreen}\chi$; but if $({\color{red}\phi}\land{\color{Orange}\psi})\phantom{\land\color{red}\phi}$ then $\color{OliveGreen}\chi$}

    }
\end{frame}

\subsection{How does contrastive stress help?}
\begin{frame}[t]
    \subsectionpage\vskip 9pt
    \begin{itemize}
        \item<2-> Our proposal: contrastive topic\vskip 4.5pt
        \begin{itemize}
            \item<3-> Establish two topics as logically disjoint and independent
        \end{itemize}
    \end{itemize}
    \visible<4->{\ex. a. What do your siblings do?\\
         b. \textbf{{\color{seeblau100!75!black}[My [SIster]\textsubscript{Focus}]\textsubscript{Topic}}} [studies MEDicine]\textsubscript{Focus}, and \textbf{{\color{seeblau100!75!black}[my [BROther]\textsubscript{Focus}]\textsubscript{Topic}}} is\\\phantom{b. }[working on a FREIGHT ship]\textsubscript{Focus}.\hfill\citep[p.~44]{Krifka2007}

}
    \visible<5->{\ex. a. What do your siblings do?\\
         b. \textbf{{\color{seeblau100!75!black}[My [BROthers]\textsubscript{Focus}]\textsubscript{Topic}}} [study MEDicine]\textsubscript{Focus}, and \#\textbf{{\color{seeblau100!75!black}[my [LITtle}}\\ \phantom{b. }\textbf{{\color{seeblau100!75!black}brother]\textsubscript{Focus}]\textsubscript{Topic}}} is [working on a FREIGHT ship]\textsubscript{Focus}.\hfill\citep[p.~267]{Krifka2007}\vskip 18pt

}
    \begin{itemize}
        \item<6-> Conditional antecedents set aboutness topic \citep{Ebert2008,Ebert2014}\vskip 4.5pt
        \item<7-> Contrastive topic gives impetus to interpret ${\color{red}\phi}$-conditional logically on its own
    \end{itemize}
\end{frame}

\subsection{Why contrastive stress on the auxiliary verb?}
\begin{frame}[t]
    \subsectionpage\vskip 9pt
    \begin{itemize}
        \item<2-> How are the auxiliary verbs semantically disjoint?\vskip 9pt
        \item<3-> Our proposal: actual target is tense-aspect-mood (TAM) morphology\vskip 4.5pt
            \begin{itemize}
                \item<4-> Antecedental TAM morphology connected to conditional's world variable
                \item<5-> TAM treated as type of bound pro-world \citep{Schlenker2005}
                \item<6-> Contrastively stressed pro-worlds behave like contrastively stressed pronouns
            \end{itemize}\vskip 4.5pt
        \item<7-> How do contrastively stressed pronouns behave?\vskip 4.5pt
            \begin{itemize}
                \item<8-> Only felicitous if the binders' domains are disjoint \citep{Sauerland1998,Sauerland1999} 
            \end{itemize}
    \end{itemize}
    \visible<9->{\ex. \resizebox{576pt}{!}{Every fourth grade boy\textsubscript{i} called {\color{seeblau100!75!black}his\textsubscript{i}} mother, but no FIFTH grade boy\textsubscript{j} called {\color{seeblau100!75!black}HIS\textsubscript{j}} mother.}

}
    \visible<10->{\ex. \resizebox{576pt}{!}{I wanted every student\textsubscript{i} to call {\color{seeblau100!75!black}his\textsubscript{i}} father, \#but only every YOUNG student\textsubscript{j} called {\color{seeblau100!75!black}HIS\textsubscript{j}} father.}

}
\end{frame}

\begin{frame}[t]
    \subsectionpage\vskip 9pt
    \begin{itemize}
        \item<1-> Formal model of this by \citet{Jacobson2004}:\vskip 4.5pt
            \begin{itemize}
                \item<2-> Bound pronoun: a partial identity function (range: its binder's domain)
            \end{itemize}
    \end{itemize}
    \visible<3->{\ex.  a. $\sem{his\textsubscript{1}}^{g,c}$ is defined if $g(1)\in D_R$ , where $D_R$ is the domain that binds the\\\phantom{a. }pronoun. When defined, $\sem{his}^{g,c} = g(1)$.\\
        b. $\sem{his\textsubscript{1}}^{g,c}={I\hspace{-0.75mm}D}_R(g(1))$\vskip 18pt

}
    \begin{itemize}
        \item<4-> Contrastive Stress:
    \end{itemize}
    \visible<5->{\ex.  a. every 3\textsuperscript{rd}-grader $[\lambda x_e.\text{call}(x, \text{the-mother-of}({\color<6->{seeblau100!75!black}I\hspace{-0.75mm}D_{\text{3\textsuperscript{rd}-graders}}}(x)))]$\\
    b. every 4\textsuperscript{th}-grader $[\lambda x_e.\text{call}(x, \text{the-mother-of}({\color<6->{seeblau100!75!black}I\hspace{-0.75mm}D_{\text{4\textsuperscript{th}-graders}}}(x)))]$

    }
\end{frame}

\begin{frame}[t]
    \subsectionpage\vskip 9pt
    \begin{itemize}
        \item Formal adoption of this by \citet{krassnig2022ReverseSobel}:\vskip 4.5pt
            \begin{itemize}
                \item<2-> Bound pro-world: a partial identity function (range: its binder's domain)
                \item<3-> Binding domain: set of worlds quantified over by conditional semantics
            \end{itemize}
    \end{itemize}
    \visible<4->{\ex.  a. $\sem{TAM\textsubscript{i}}^{g,c}$ is defined if $g(i)\in D_R$ , where $D_R$ is the domain that binds the\\\phantom{a. }pro-world. When defined, $\sem{TAM\textsubscript{i}}^{g,c} = g(i)$.\\
        b. $\sem{TAM\textsubscript{i}}^{g,c}={I\hspace{-0.75mm}D}_R(g(i))$\vskip 18pt

}
    \begin{itemize}
        \item<5-> Contrastive Stress:
    \end{itemize}
    \visible<6->{\ex.  a. If $[\lambda w_s .{\color{red}\phi({\color<7->{seeblau100!75!black}I\hspace{-0.75mm}D_{\text{Domain-A}}}(w))} \land {\color{Orange}\psi({I\hspace{-0.75mm}D_{\text{Domain-A}}}(w))}]$, (then) $[\lambda w_s.\neg {\color{OliveGreen}\chi(w)}]$\\
    b. If $[\lambda w_s .{\color{red}\phi({\color<7->{seeblau100!75!black}I\hspace{-0.75mm}D_{\text{Domain-B}}}(w))}]$, (then) $[\lambda w_s. {\color{OliveGreen}\chi(w)}]$

    }
\end{frame}

\begin{frame}[t]
    \subsectionpage\vskip 9pt
    \begin{itemize}
        \item What do Domain-A and Domain-B correspond to?\vskip 9pt
            \begin{itemize}
                \item<2-> Dynamic Strict: modal horizon VS modal horizon shrunk to closest {\color{red}$\phi$}-worlds
                \item<3-> Variably-Strict: closest ${\color{red}\phi}\land{\color{Orange}\psi}$-worlds VS closest ${\color{red}\phi}$-worlds (no subordination)
            \end{itemize}\vskip 18pt
        \item<4-> rSS are only felicitous if ${I\hspace{-0.75mm}D_{\text{Domain-A}}}\cap{I\hspace{-0.75mm}D_{\text{Domain-B}}}=\emptyset$\vskip 9pt
            \begin{itemize}
                \item<5-> rendering the contrastive topic successful
                \item<6-> permanently shrinking the modal horizon or cancelling modal subordination
                \item<7-> which guarantees a non-contradictory sequence
            \end{itemize}\vskip 18pt
        \item<8-> What would the two approaches predict for counterfactual acausal rSS?
    \end{itemize}
\end{frame}

\begin{frame}[t]
	\subsectionpage\vskip 9pt
	\begin{itemize}
        \item<1->	Each deviance to $w_0$ decreases world similarity
	\end{itemize}\vspace{-5mm}
	\visible<2->{
	\begin{figure}[ht!]
\centering
\input{tikz/acausal}
\label{fig:acausal}
\end{figure}}\vspace{-7.5mm}
	\begin{itemize}
        \item<4->  \resizebox{608pt}{!}{Either approach renders counterfactual acausal rSS felicitous (${I\hspace{-0.75mm}D_{\text{Domain-A}}}\cap{I\hspace{-0.75mm}D_{\text{Domain-B}}}=\emptyset$)}
	\end{itemize}
\end{frame}

\setcounter{subsection}{2}
\subsection*{Mid-Summary}
\begin{frame}[t]
\subsectionpage\vskip 9pt
    \begin{itemize}
        \item<1-> Questions to be answered:\vskip 18pt
            \begin{enumerate}
                \item Why are rSS without contrastive stress infelicitous? {\color{seeblau100!75!black}\checkmark}\vskip 9pt
                \item How does contrastive stress help? {\color{seeblau100!75!black}\checkmark}\vskip 9pt
                \item Why contrastive stress on auxiliary verb? {\color{seeblau100!75!black}\checkmark}\vskip 9pt
                \item How does a causal link between {\color{red}$\phi$} and {\color{Orange}$\psi$} cause infelicity?\vskip 9pt
                \item How does non-counterfactuality cause infelicity?\vskip 9pt
                \item How does the epistemic dismissal of {\color{Orange}$\psi$} rescue rSS?
            \end{enumerate}
	\end{itemize}
\end{frame}

\subsection{How does a causal link between $\phi$ and $\psi$ cause infelicity?}
\begin{frame}<1-3,6->[t]
\subsectionpage\vskip 9pt
    \begin{itemize}
        \item<2-> \citet{Bennett2003} \&\ \citet{Arregui2009}: only cause-initial deviances decrease world similarity\vskip 4.5pt
            \begin{itemize}
                \item<3-> Independently motivated\vskip 4.5pt
                \item<6-> If ${\color{Orange}\psi}$ is causally preceded by ${\color{red}\phi}$: closest ${\color{red}\phi}\land{\color{Orange}\psi}$-worlds $\subseteq$ closest ${\color{red}\phi}$-worlds
                \item<7-> If ${\color{Orange}\psi}$ is not causally preceded by ${\color{red}\phi}$: closest ${\color{red}\phi}\land{\color{Orange}\psi}$-worlds $\not\subseteq$ closest ${\color{red}\phi}$-worlds
            \end{itemize}\vskip 18pt
        \item<8-> What impact would this have on our analysis of causal rSS?
    \end{itemize}
\end{frame}

\begin{frame}[t]
	\subsectionpage\vskip 9pt
	\begin{itemize}
        \item<1->	Only the cause-initial ${\color{red}\phi}$ deviance to $w_0$ decreases world similarity
	\end{itemize}\vspace{-5mm}
	\begin{figure}[ht!]
\centering
\begin{tikzpicture}\visible<2->{
	\coordinate (O) at (0,0);

	\draw[fill=white] (O) circle (2);
    \fill[seeblau100,rotate=90] (O) + (0, -1.2) arc (270:450:1.2);
    \draw[color=red,line width=1mm] (O) circle (1.2);
	\draw[fill=white] (O) circle (0.4);
    \draw[fill=white,line width=0mm,white] (0.6,-0.5) circle (0.325);
    \draw[fill=white,line width=0mm,white] (-0.6,-0.5) circle (0.325);

    \node at (0,0) {w$_0$};
	\node at (0,0.7) {w$_1$};
	\node at (0.6,-0.5) {w$_2$};
	\node at (-0.6,-0.5) {w$_3$};
	
	\node at (0,-1.6) {w$_6$};
	\node at (-1,1.2) {w$_4$};
	\node at (1,1.2) {w$_5$};
	
	\node at (0,-2.5) {If ${\color{red}\phi}\land{\color{Orange}\psi}$, then not $\color{OliveGreen}\chi$};
    \node at (0,-3.15) {\small (Domain-A)};
	
	\draw[color=seeblau100,line width=2mm,>={Triangle[length=4mm,width=4mm]},->] (2.5,0) -- (3.5,0);

    \node at (3,2.5) {\textbf{Dynamic Strict}};
	
	\begin{scope}[xshift=6cm]
	\coordinate (O) at (0,0);

	\draw[fill=white] (O) circle (2);
	\draw[color=red,fill=seeblau100,line width=1mm] (O) circle (1.2);
	\draw[fill=white] (O) circle (0.4);

    \node at (0,0) {w$_0$};
	\node at (0,0.7) {w$_1$};
	\node at (0.6,-0.5) {w$_2$};
	\node at (-0.6,-0.5) {w$_3$};
	
	\node at (0,-1.6) {w$_6$};
	\node at (-1,1.2) {w$_4$};
	\node at (1,1.2) {w$_5$};
	
	\node at (0,-2.5) {If $\color{red}\phi$, then $\color{OliveGreen}\chi$};	
 \node at (0,-3.15) {\small (Domain-B)};}
  \visible<3->{
	\node at (3,0) {\textbf{VS}};
	
	\begin{scope}[xshift=6cm]
	\coordinate (O) at (0,0);

	\draw[fill=white] (O) circle (2);
    \fill[seeblau100,rotate=90] (O) + (0, -1.2) arc (270:450:1.2);
    \draw[color=black] (O) circle (1.2);
	\draw[fill=white] (O) circle (0.4);
    \draw[fill=white,line width=0mm,white] (0.6,-0.5) circle (0.325);
    \draw[fill=white,line width=0mm,white] (-0.6,-0.5) circle (0.325);

    \node at (0,0) {w$_0$};
	\node at (0,0.7) {w$_1$};
	\node at (0.6,-0.5) {w$_2$};
	\node at (-0.6,-0.5) {w$_3$};
	
	\node at (0,-1.6) {w$_6$};
	\node at (-1,1.2) {w$_4$};
	\node at (1,1.2) {w$_5$};
	
	\node at (0,-2.5) {If ${\color{red}\phi}\land{\color{Orange}\psi}$, then not $\color{OliveGreen}\chi$};
 \node at (0,-3.15) {\small (Domain-A)};
	
	\draw[color=seeblau100,line width=2mm,>={Triangle[length=4mm,width=4mm]},->] (2.5,0) -- (3.5,0);

    \node at (3,2.5) {\textbf{Variably-Strict}};
	
	\begin{scope}[xshift=6cm]
	\coordinate (O) at (0,0);

	\draw[fill=white] (O) circle (2);
	\draw[fill=seeblau100] (O) circle (1.2);
	\draw[fill=white] (O) circle (0.4);

    \node at (0,0) {w$_0$};
	\node at (0,0.7) {w$_1$};
	\node at (0.6,-0.5) {w$_2$};
	\node at (-0.6,-0.5) {w$_3$};
	
	\node at (0,-1.6) {w$_6$};
	\node at (-1,1.2) {w$_4$};
	\node at (1,1.2) {w$_5$};
	
	\node at (0,-2.5) {If $\color{red}\phi$, then $\color{OliveGreen}\chi$};
 \node at (0,-3.15) {\small (Domain-B)};
	\end{scope}
	\end{scope}
	\end{scope}}
\end{tikzpicture}
\label{fig:causal}
\end{figure}\vspace{-7.5mm}
	\begin{itemize}
        \item<4->  \resizebox{608pt}{!}{Either approach renders counterfactual causal rSS infelicitous (${I\hspace{-0.75mm}D_{\text{Domain-A}}}\cap{I\hspace{-0.75mm}D_{\text{Domain-B}}}\neq\emptyset$)}
	\end{itemize}
\end{frame}

\subsection{How does non-counterfactuality cause infelicity?}
\begin{frame}[t]
\subsectionpage\vskip 9pt
    \begin{itemize}
        \item<2-> \citet{krassnig2022ReverseSobel}: all non-CF worlds (must) occupy the same sphere of similarity\vskip 4.5pt
            \begin{itemize}
                \item<3-> Not a rare position, albeit typically differently formulated\vskip 4.5pt
                \item<4-> \citet{Lewis1973}: Variably-strict semantics only apply to counterfactuals
                \item<5-> \citet{Lewis1973}: Non-counterfactuals might use strict semantics\vskip 4.5pt
                \item<6-> Functionally equivalent to our proposal\vskip 9pt
                \item<7-> Others: all non-CF worlds are quantified over by a probabilistic semantics \citep{adams1966probability,Edgington1995,Berto2021}
            \end{itemize}\vskip 18pt
        \item<8-> What impact would this have on our analysis of non-CF rSS?
    \end{itemize}
\end{frame}

\begin{frame}[t]
	\subsectionpage\vskip 9pt
	\begin{itemize}
        \item<1->	Non-CF ${\color{red}\phi}$-worlds and non-CF ${\color{red}\phi}\land{\color{Orange}\psi}$-worlds occupy same degree of similarity
	\end{itemize}\vspace{-5mm}
	\begin{figure}[ht!]
\centering
\input{tikz/noncf}
\label{fig:noncf}
\end{figure}\vspace{-7.5mm}
	\begin{itemize}
        \item<4->  \resizebox{608pt}{!}{Either approach renders non-counterfactual rSS infelicitous (${I\hspace{-0.75mm}D_{\text{Domain-A}}}\cap{I\hspace{-0.75mm}D_{\text{Domain-B}}}\neq\emptyset$)}
	\end{itemize}
\end{frame}

\subsection{How does the epistemic dismissal of $\psi$ rescue rSS?}
\begin{frame}[t]
    \subsectionpage\vskip 9pt
    \begin{itemize}
        \item<2-> \citet{krassnig2022ReverseSobel}: Epistemic exclusion of ${\color{Orange}\psi}$ removes all ${\color{Orange}\psi}$ in quantificational domain\vskip 9pt
            \begin{itemize}
                \item<3-> Exception: if exclusion renders quantificational domain empty
                \item<4-> Simple case of quantificational domain restriction \citep[amongst many others:][]{Fintel1994,Reimer1998,Stanley2000}\vskip 4.5pt
                \item<5-> Epistemic exclusion can be covert from the beginning, or \ldots
                \item<6-> overtly enforced inbetween sequence conditionals
            \end{itemize}\vskip 18pt
        \item<7-> What impact would this have on our analysis?\vskip 9pt
            \begin{itemize}
                \item<8-> ${\color{red}\phi}\land{\color{Orange}\psi}$-conditionals behave normally (domain would be empty otherwise)
                \item<9-> ${\color{red}\phi}$-conditionals quantify over $D_{\text{Domain-B}}\setminus D_{\color{Orange}\psi}$
            \end{itemize}
    \end{itemize}
\end{frame}

\begin{frame}[t]
	\subsectionpage\vskip 9pt
	\begin{itemize}
        \item<1->	Non-counterfactual causal rSS exclude ${\color{Orange}\psi}$-worlds for second conditional
	\end{itemize}\vspace{-5mm}
	\begin{figure}[ht!]
\centering
\begin{tikzpicture}\visible<2->{
	\coordinate (O) at (0,0);

    \draw[fill=white] (O) circle (2);
    \fill[seeblau100] (O) + (0, -1.2) arc (270:450:1.2);
	\draw[line width=1mm,color=red] (O) circle (1.2);
    \draw[fill=white,white] (O) circle (0.4);

    \node at (0,0) {w$_0$};
	\node at (-0.6,0.5) {w$_1$};
	\node at (0.6,0.5) {w$_2$};
	\node at (-0.6,-0.5) {w$_3$};
    \node at (0.6,-0.5) {w$_4$};
	
	\node at (0,-1.6) {w$_7$};
	\node at (-1,1.2) {w$_5$};
	\node at (1,1.2) {w$_6$};
	
	\node at (0,-2.5) {If ${\color{red}\phi}\land{\color{Orange}\psi}$, then not $\color{OliveGreen}\chi$};
    \node at (0,-3.15) {\small (Domain-A)};
	
	\draw[color=seeblau100,line width=2mm,>={Triangle[length=4mm,width=4mm]},->] (2.5,0) -- (3.5,0);

    \node at (3,2.5) {\textbf{Dynamic Strict}};
	
	\begin{scope}[xshift=6cm]
	\coordinate (O) at (0,0);

    \draw[fill=white] (O) circle (2);
    \fill[seeblau100] (O) + (0, 1.2) arc (90:360:1.2);
    \fill[white] (O) + (0, -1.2) arc (270:450:1.2);
	\draw[line width=1mm,color=red] (O) circle (1.2);
    \draw[fill=seeblau100,seeblau100] (O) circle (0.4);

    \node at (0,0) {w$_0$};
	\node at (-0.6,0.5) {w$_1$};
	\node at (-0.6,-0.5) {w$_3$};
	
	\node at (0,-1.6) {w$_7$};
	\node at (-1,1.2) {w$_5$};
	\node at (1,1.2) {w$_6$};
	
	\node at (0,-2.5) {If $\color{red}\phi$, then $\color{OliveGreen}\chi$};	
 \node at (0,-3.15) {\small (Domain-B)};}
  \visible<3->{
	\node at (3,0) {\textbf{VS}};
	
	\begin{scope}[xshift=6cm]
	\coordinate (O) at (0,0);

    \draw[fill=white] (O) circle (2);
    \fill[seeblau100] (O) + (0, -1.2) arc (270:450:1.2);
	\draw (O) circle (1.2);
    \draw[fill=white,white] (O) circle (0.4);

    \node at (0,0) {w$_0$};
	\node at (-0.6,0.5) {w$_1$};
	\node at (0.6,0.5) {w$_2$};
	\node at (-0.6,-0.5) {w$_3$};
    \node at (0.6,-0.5) {w$_4$};
	
	\node at (0,-1.6) {w$_7$};
	\node at (-1,1.2) {w$_5$};
	\node at (1,1.2) {w$_6$};
	
	\node at (0,-2.5) {If ${\color{red}\phi}\land{\color{Orange}\psi}$, then not $\color{OliveGreen}\chi$};
 \node at (0,-3.15) {\small (Domain-A)};
	
	\draw[color=seeblau100,line width=2mm,>={Triangle[length=4mm,width=4mm]},->] (2.5,0) -- (3.5,0);

    \node at (3,2.5) {\textbf{Variably-Strict}};
	
	\begin{scope}[xshift=6cm]
	\coordinate (O) at (0,0);

	\draw[fill=white] (O) circle (2);
    \fill[seeblau100] (O) + (0, 1.2) arc (90:360:1.2);
    \fill[white] (O) + (0, -1.2) arc (270:450:1.2);
	\draw (O) circle (1.2);
    \draw[fill=seeblau100,seeblau100] (O) circle (0.4);

    \node at (0,0) {w$_0$};
	\node at (-0.6,0.5) {w$_1$};
	\node at (-0.6,-0.5) {w$_3$};
	
	\node at (0,-1.6) {w$_7$};
	\node at (-1,1.2) {w$_5$};
	\node at (1,1.2) {w$_6$};
	
	\node at (0,-2.5) {If $\color{red}\phi$, then $\color{OliveGreen}\chi$};
 \node at (0,-3.15) {\small (Domain-B)};
	\end{scope}
	\end{scope}
	\end{scope}}
\end{tikzpicture}
\label{fig:exclusion}
\end{figure}\vspace{-7.5mm}
	\begin{itemize}
        \item<4->  \resizebox{608pt}{!}{Either approach necessarily renders such rSS felicitous (${I\hspace{-0.75mm}D_{\text{Domain-A}}}\cap{I\hspace{-0.75mm}D_{\text{Domain-B}}}=\emptyset$)}
	\end{itemize}
\end{frame}